
3 Spatial disparity-coherent watermarking :

As said in the previous chapter a number of articles focused on how to incorporate depth information into the perceptual shaping process of the embedded watermark. This process allows to achive disparity-coherence and makes sure that a physical point of the captured scene carries the same watermark sample regardless of where it appears in the left and right view.

This process brings two advantages: it produces stereoscopic views more in line with reality therefore yields less visual discomfort; and is expected to have superior robustness against view synthesis.

3.1 Prior work

A prior work that's based on the disparity-coherent technique is the one carried on by Doerr et al in "Blind Detection for Disparity-Coherent
Stereo Video Watermarking" []. 
The watermark strategy  assume that the key-seeded reference watermark pattern w K ∼ N (0, 1) is embedded spatially in the left view and subsequently transferred to the right one.
The watermark embedding and detection operations for the left view are therefore given by the conventional spread-spectrum equations:
FORMULA LEFT 
where the superscript (w) indicates watermarked quantities, the subscript L (resp. R ) denotes quantities related
to the left (resp. right) view, α > 0 is the embedding strength, and w L is normally distributed with zero mean and unit variance.
The embedding strenght used to keep the embedding distortion impercep-
tible is alpha=3. 

For the right view, the watermarking equation is the same, except that the watermark pattern w K is warped according to the depth information prior to insertion.
FORMULA RIGHT
The watermark detection on the right view relies on the computation of a horizontal cross-correlation array. 
FORMULA
The correlation array is then mapped into a scalar value in order to compare it with a threshold and decide whether the tested content contains the watermark, three possible mapping functions are proposed:
FORMULA


3.2 Gaussian-noise disparity-coherent watermarking

Based on the described technique a spatial watermarking technique is proposed.
For the spatial watermark its been taken under consideration the insertion of a Gaussian-noise reference watermark in an additive way.
As in Doerr et al, the left view is processed in the conventional way, with spred-spectrum equations (riferimento all'equazione); the watermark is then warped according to the disparity value and inserted in the right view (rif all'eq), taking under consideration that the occluded zones shoudn't be processed.
 The added pattern and the reference images have the same size, so it should be noted that the warping process will generate a loss of marked pixel, due to the baseline's lenght.
(IMMAGINI DI DISP E OCC MAP, E CALCOLATE CON KZ)
Since the disparity map and the occclusion map are usually not available, it needs to be estimated through the KZ algorithm, before the warping process.
The embedding strenght is alpha=1; it should be noted that this baseline watermarking framework could be enriched with conventional add-ons, e.g. perceptually modulate the embedding strength to better accommodate
for the human visual system or canceling host interference for improved detection statistics.

In the detection process, its been used a conventional correlation-based detector for the left view (ref to eq). 
On the other hand to detect the watermark in the right view two differet correlation-based strategies are proposed:
in the first strategy its computed the correlation value between the non-distorted watermark and the right view warped according to the right-to-left disparity: this way the priviously warped watermark is restored, even if there will be discontinuities due the occluded zones.
FORMULA
The second strategy is again a simple correlation-based detector, but the correlation value is computed between the right view and the warped watermark instead of the original one.
FORMULA  

Has said before the disparity-coherent watermarking have the ability to detect the embedded watermark in synthetized views: to performe the detection on a random right view, that might be synthetized, the detector will need to calculate the disparity map between the analyzed view and the recieved left, and warp it accordingly, to recompose the original watermark. 
There is then a tight bond between the watermarking process and the evaluation of the disparity maps; with the graph-cuts algorithm its possible to compute accurate maps and to know the occluded zones in a Non-real time way. 
