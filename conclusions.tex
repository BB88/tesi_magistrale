\chapter{Conclusions}
\markright{Conclusions}
\label{concl}
\phantomsection
%\addcontentsline{toc}{chapter}{Conclusions}

In this thesis a disparity-coherent watermarking algorithm has been implemented. It works in the frequency and spatial domain: a pseudo-random sequence of real numbers is embedded in a selected set of DFT coefficients of the left image then the reference watermark is spatially inserted in a disparity-coherent way in the right view.\\
The method has shown good results in quality measure tests and roubustness test against view synthesis and compression.\\
To evaluate the quality of the watermarked video sequence PSNR value and new measures based on SSIM value have been used. The experimental results shows that ..... (discorso trasparenza e robustezza, mettendo anche i valori massimi raggiunti dal PSNR e dal QM).
Future works can provide a visual mask to improve the quality of the waterkmarked video sequence, and add a synchronization pattern to make the watermak robust against geometrical attacks.



