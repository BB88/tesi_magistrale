\chapter{Conclusions}
\markright{Conclusions}
\label{concl}
\phantomsection
%\addcontentsline{toc}{chapter}{Conclusions}

In this thesis a blind disparity-coherent watermarking algorithm has been implemented. While prior works only inserted the mark in the spatial domain, in this case both the frequency and spatial domains are considered. \\
The marking process can be summarized in two steps: (i) a pseudo-random sequence of real numbers is embedded in a selected set of DFT coefficients of the left image, (ii) the reference watermark is spatially inserted in a disparity-coherent way in the right view.\\ 
A new detection process is then proposed, which is also based on the disparity map: (i) the detection on the left view is performed according to a criterion based on statistical decision theory; (ii) the detection on the right view is performed by first warping it according to the right-to-left disparity, in order to resynchronize it to its initial shape, and then the previous criterion is applied.
 
The method has been tested against compression attacks and web uploading. It emerged that the watermark can resist until a compression with constant rate factor equal to 25 with good detection statistics; besides, since the mark become less visible the more the compression rate increases, it is possible to employa higher embedding power in case the content has to be heavily compressed.\\

The method has also proved to be robust against view synthesis, thanks to the fact that the detection process resynchronizes the watermark on the right view before performing the detection.

To evaluate the quality of the watermarked video sequence PSNR value and new measures based on SSIM value have been used. The experimental results show that the video quality degrades inversely with the power of the watermark, but it mantains a good measure when marking with power lower than 0.6; PSNR reaches a maximum of  $44.438$ dB with a power value of $0.3$.\\

Future works could concern the study of a visual mask to improve the quality of the waterkmarked video sequence; the investigation in deeper detail of the robustness to the processing applied by social network applications like Youtube, and the introduction of a synchronization pattern to make the watermak robust against geometrical attacks. 

Further investigations are also needed to better comprehend the sensitivity of the human eye to noise addition in the left and right view.



