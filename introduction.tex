\chapter*{Introduction}
\markright{Introduction}
\label{intro}
\phantomsection
\addcontentsline{toc}{chapter}{Introduction}

In the last few years stereoscopy has become a great part of image and video processing.\\
In medical diagnosis and endoscopic surgery \cite{MED}\cite{MED2} as in fault detection in manufactory industry, army and arts,
multiview imaging is considered as a key enabler  for professional added value services.\\
Nowdays stereoscopic techniques are also used in people tracking \cite{TRACK} and mobile robotics
navigation \cite{PG} for economic reasons and to improve performances.\\
Finally the worldwide success of 3D movie releases and 3D video games \cite{GAME} and the deployment of 3D televisions made the nonprofessional user aware about a new type of multimedia entertainment experience.\\
The increasing production and distribution of these contents leads to the concerns over copyright protection.\\
Digital watermarking can be considered as a powerful property right protection technology, since it adds some information (a mark, i.e. copyright information) in the
original content without altering its visual quality; in this way such a marked content can be further distributed/consumed by another user without any restriction; still, the legitimate/illegitimate usage can be determined at any moment by detecting the mark. At the same time, the watermarking protection mechanism, instead of restricting the media copy/distribution/consumption, provides means for tracking the source of the content illegitimate usage.\\
The purpose of this thesis is to provide a new watermarking system for copyright protection of stereoscopic videos.\\
The method operates in the frequency and in the spatial domain by embedding a pseudo-random sequence of real numbers in a selected set of DFT coefficients of the left image; then the reference watermark is distorted according to the depth information prior to insertion and spatially added to the right image.\\
Thanks to this embedding procedure, the watermark is robust against view synthesis and lossy compression.
The thesis is structured as follows: in Chapter \ref{stereo_video} the stereoscopic video context is presented, specifically the devices used to capture the scene and to display it, and the stereoscopic vision background.\\
In Chapter \ref{wat} an overview of the digital watermarking process is presented.\\
Chapter \ref{spa} and \ref{dft} present a new correlation-based detection for spatial disparity-coherent watermaking technique and a new disparity-coherent watermarking techinique which works in the frequency domain, respectively.\\
Finally, in Chapter \ref{exp} the experimental results conducted on the proposed algorithms are presented.\\


